\documentclass[%
	11pt,
	a4paper,
	utf8,
	%twocolumn
		]{article}	

\usepackage{style_packages/podvoyskiy_article_extended}


\begin{document}
\title{Заметки по прогнозированию временных рядов}

\author{}

\date{}
\maketitle

\thispagestyle{fancy}

\tableofcontents

\section{Вводные замечания}

Возможно самой простой формой прогнозирования является \emph{скользящее среднее}. Часто скользящее среднее используется в качестве \emph{метода сглаживания}, чтобы найти более гладкую линию для данных с большим количеством вариаций \cite{gruzdev:time-series-2022}. Каждую точку данных можно описать средним значением $ n $ окружающих точек данных, где $ n $ -- обозначает размер окна. При размере окна 10 мы опишем точку данных так, чтобы ее значения представляло собой среднее значение 5 значений, предшествующих точке, и 5 значений, следующих после точки. При прогнозировании будущие значения рассчитываются как среднее $n$ предыдущих значений, поэтому при размере окна 10 речь будет идти о среднем значении 10 предыдущих значений.

Суть сглаживания скользящим средним заключается в том, что вам нужен большой размер окна, чтобы сгладить шум и уловить фактический тренд, но при большом размере окна ваши прогнозы будут значительно отставать от тренда, поскольку вы будете использовать все более ранние наблюдения для вычисления среднего.

Идея экспоненциального сглаживания заключается в том, что мы применяем \emph{экспоненциально уменьшающиеся веса} к усредняемым по времени значениям, придавая недавним значениям больший вес, а большее ранним значениям -- меньший.

Хольт усовершенствовал технику экспоненциального сглаживания, чтобы она позволяла учитывать тренд, и назвал ее \emph{двойным экспоненциальным сглаживанием} (double exponential smoothing). А в сотрудничестве с Винтерсом Хольт добавил поддержку сезонности в 1960 году, и техника получала название \emph{тройного экспоненциального сглаживания} (экспоненциальное сглаживание Хольта-Винтерса).

Недостатком этих методов прогнозирования является то, что \emph{\color{red}они могут медленно приспосабливаться к новым тенденциям}, и поэтому прогнозируемые значения отстают от реальности -- \emph{они плохо работают, когда требуется более длительные горизонты прогнозирования}, и существует мнрожество гиперпараметров для настройки, что может быть трудным и очень времязатратным процессом \cite[\strbook{14}]{gruzdev:time-series-2022}.

ARIMA и модель Бокса-Дженкинса часто используется как вазимозаменяемые термины, хотя технически модель Бокса-Дженкинса относится к методу оптимизации параметров для ARIMA-модели.

ARIMA -- это аббривиатура трех понятий: \emph{автогрессия}, \emph{интегрированное} и \emph{скользящее среднее}. \emph{Авторегрессия} означает, что модель использует зависимость между точкой данных и некоторым количеством запаздывающих точек данных (лагов). То есть модель делает прогноы на основе предыдущих значений. Это похоже на предсказание того, что завтра будет тепло, потому что до сих пор было тепло всю неделю.

Интегрированное означает, что вместо применения любой исходной точки данных используется разница между этой точкой данных и некоторой предыдущей точкой данных. По сути, это означает, что мы преобразуем ряд значений в ряд изменений значений. Интуитивно это предпалагает, что завтра будет более или менее такая же температура, как сегодня, потому что температура всю неделю сильно не менялась.

Проблема с ARIMA-моделями заключается в том, что они не поддерживают сезонность или данные с повторяющимися циклами, такими как повышение температуры днем и падение ночью или повышение летом и падение зимой. SARIMA (Seasonal ARIMA) бала разработана для преодоления данного недостатка. Подобно нотации ARIMA, нотация для модели SARIMA представляет собой $SARIMA(p,d,q)(P, D, Q)m$, где $P$ -- порядок сезонной авторегрессии, $D$ -- порядок сезонного дифференциирования, $Q$ -- порядок сезонного скользящего среднего, а $m$ -- периодичность (количество периодов в полном сезонном цикле).

VARIMA для случаев с несколькими временными рядами в качестве векторов. FARIMA -- дробная ARIMA и ARFIMA -- дробная проинтегрированная ARIMA. Последние две включают дробную степень дифференциирования, обеспечивающую длительную память в том смысле, что наблюдения, удаленные друг от друга с точки зрения времени, могут иметь несущественные зависимости.

SARIMAX -- сезонная ARIMA, где X означает экзогенные или дополнительные переменные, добавленные в модель, например добавление прогнгоза осадков в модель прогнозирования температуры. 

ARIMA обычно показывает очень хорошие результаты, но недостатками являются сложность подбора параметров и необходимость тщательного разведочного анализа. Настройка и оптимизация моделей ARIMA часто требуют значительных вычислительных ресурсов, а успешные результаты могут зависеть от навыков и опыта прогнозиста. 

Когда дисперсия данных не является постоянной во времени, ARIMA-модели сталкиваются с проблемами \cite[\strbook{16}]{gruzdev:time-series-2022}. В экономике и финансах непостоянство дисперсии может быть обычным являением. Для решения этой проблемы были разработаны модели \emph{авторегрессии условной гетероскедостичности} (Autoregressive Conditional Heteroscedasticity -- ARCH). Гетероскедостичность -- это способ сказать, что дисперсия или разброс данных не являются постоянными повсюду, а противоположенным термином является гомоскедостичность.

Тим Боллерслев и Стивен Тейлор в 1986 году дополнили модель ARCH компонентой \emph{скользящего среднего}, предложив модель Generalized ARCH, или GARCH. Когда колебания случайны, может быть полезна GARCH.

{\color{red}Обе модели ARCH и GARCH не могут обрабатывать ни тренд, ни сезонность}, хотя на практике часто для работы с \emph{сезонными колебаниями} и \emph{трендом} временного ряда применяется ARIMA-модель, а затем для моделирования \emph{ожидаемой дисперсии} используется ARCH-модель.

Градиентный бустинг сейчас стал популярен для прогнозирования временных рядов. Следует помнить, что {\color{red}деревья не умеют экстраполировать!} Если временной ряд содержит тренд, можно попробовать детрендинг. Мы удаляем из ряда тренд, предварительно спрогнозированный линейной моделью. На полученных остатках обучаем градиентный бустинг и получаем прогнозы, к ним добавляем тренд. Преимуществом градиентного бустинга является способность фиксировать нелинейные зависимости и взаимодействия высокого порядка.

\section{Математический аппарат Facebook Prophet}

\subsection{Библиотека Facebook Prophet}

Используется модель разложимых временных рядов (Harvey \& Peters, 1990) с тремя основными компонентами:
\begin{itemize}
	\item тренд,
	
	\item сезонность,
	
	\item праздники
\end{itemize}
\begin{align*}
y(t) = g(t) + s(t) + h(t) + \varepsilon_t,
\end{align*}
где $g(t)$ -- функция тренда, моделирующая непериодические изменения значений временного ряда, $s(t)$ представляет собой периодические изменения (например, еженедельную и ежегодную сезонность), а $h(t)$ -- представляет собой эффекты праздников, которые возникают в течение одного или нескольких дней. Член ошибки $\varepsilon_t$ представляет собой любые случайные возмущения, которые не учитываются моделью. 









% Источники в "Газовой промышленности" нумеруются по мере упоминания 
\begin{thebibliography}{99}\addcontentsline{toc}{section}{Список литературы}
    \bibitem{mckinney:pandas-2015}{\emph{Маккинли У.} Python и анализ данных, 2015. -- 482 с.}
	
	\bibitem{gruzdev:time-series-2022}{\emph{Груздев А.} Прогнозирование временных рядов с помощью Facebook, Prophet, ETNA, sktime и LinkedIn Greykite: Строим, настраиваем, улучшаем модели прогнозирования временных рядов с помощью специальных библиотек. -- М.: ДМК Пресс, 2023. -- 780 с.}
\end{thebibliography}

%\listoffigures\addcontentsline{toc}{section}{Список иллюстраций}

%\lstlistoflistings\addcontentsline{toc}{section}{Список листингов}

\end{document}
